\documentclass{article}
\usepackage{polski}
\usepackage[utf8]{inputenc}
\usepackage{lastpage}
\usepackage{listings}
\usepackage{natbib}
\usepackage{graphicx}
\usepackage{fancyhdr}
\usepackage{tikz}
\usepackage[normalem]{ulem}
\usepackage{hyperref}

\useunder{\uline}{\ul}{}

\pagestyle{fancy}
\fancyhf{}
\fancyhead[LE,RO]{Michał Żdanuk}
\fancyhead[RE,LO]{Specyfikacja funkcjonalna: \texttt{Kwantowe Kółko i Krzyżyk} }
\fancyfoot[CE,CO]{\leftmark}
\fancyfoot[LE,RO]{\thepage}

\renewcommand{\headrulewidth}{2pt}
\renewcommand{\footrulewidth}{1pt}
\rfoot{Strona \thepage \hspace{1pt} z \pageref{LastPage}}



\begin{document}

\begin{titlepage}
   \begin{center}
        \vspace*{1cm}
        \Large    
        \textbf{Specyfikacja funkcjonalna:\\ \texttt{Kwantowe Kółko i Krzyżyk}} 
            
        \vspace{1.5cm}

       \textbf{Michał Żdanuk}

       \vfill
            
       \vspace{0.8cm}
     
       \includegraphics[width=0.4\textwidth]{pw}
            
       Wydział Elektryczny\\
       Politechnika Warszawska\\
       Marzec 2022\\
       \vspace{0.8cm}
        \small

   \end{center}
   
\end{titlepage}
\newpage
\setcounter{page}{2}
\section{Cel dokumentu}
Dokument powstał, by przybliżyć czytelnikowi działanie gry - \textbf{Kwantowe Kółko i Krzyżyk}. Przedstawia on także zasady rozgrywki oraz wyjaśnia zachowanie programu w zależności od działań podejmowanych przez graczy. \\

Poza tym dokument zawiera: 
\begin{itemize}
%poprawic
    \item szablon interfejsu graficznego okna rozgrywki z uwzględnionymi komunikatami (błędu/informacyjnymi),
    \item schemat graficzny przykładowej planszy,
    \item krótki scenariusz poparty grafikami mający na celu przybliżenie zasad gry,
\end{itemize}

\section{Cel projektu}

Celem projektu jest napisanie programu w języku JAVA, pozwalającego na przeprowadzenie rozgrywki w okienku terminala dla dwóch graczy. W przypadku sprawnego zaimplementowania wersji "konsolowej" gra zostanie rozszerzona o możliwość zagrania w wersji "okienkowej" (tzn. z interfejsem graficznym).\\

Rozgrywka w \textbf{Kwantowym Kółku i Krzyżyk} będzie przebiegać na jednym urządzeniu stacjonarnym. Gracz wybiera i zaznacza swoim znakiem odpowiednią komórkę poprzez podanie do terminalu cyfry odpowiadającej danej komórce.\\

Kluczowa będzie dla mnie praca z uwzględnieniem norm programowania obiektowego tzn. poprawne korzystanie z dziedziczenia klas, agregacji czy kompozycji. Wysoki nacisk położę również na właściwe wykorzystanie podstawowych zasad obiektowości: hermetyzacji interfejsów, abstrakcji oraz hierarchizacji obiektów. 

\newpage
\section{Wstęp teoretyczny}

\subsection{Zasady rozgrywki}
\textbf{Gra Kwantowe Kółko i Krzyżyk} to uogólnienie "kwantowe" standardowej gry \texttt{Kółko i Krzyżyk}, w której ruchy graczy są "superpozycjami" ruchów z klasycznego wariantu gry. Ułożenie odpowiednich trzech znaków w rzędzie: pionowo, poziomo lub po skosie zapewnia graczowi zwycięstwo w rozgrwywce.

\subsection{Wynalazcy gry}
Za wymyślenie gry w kwantowe kółko i krzyżyk odpowiadają: \texttt{Allan Goff} oraz \texttt{Novatia Labs}.\\
Określili oni grę jako sposób "wprowadzenia fizyki kwantowej bez użycia matematyki" równocześnie tworząc narzędzie pozwalające na bardzo podstawowe zrozumienie idei mechaniki kwantowej.

\subsection{Podłoże fizyki kwantowej w rozgrywce}
Twórcy tworząc kwantowe kółko i krzyżyk chcieli umieścić w rozgrywce koncept trzech zjawisk układów kwantowych:
\begin{itemize}
    \item \textbf{superpozycję (ang. superposition)} - zdolność obiektów kwantowych do bycia w dwóch miejscach jednocześnie;
    \item \textbf{splątanie (ang. entanglement)} - zjawisko, w którym odległe obiekty systemu kwantowego wykazują korelację, której nie można wytłumaczyć żadną zależnością przyczynowo-skutkową;
    \item \textbf{zawalanie się/rozłączanie (ang. collapse)} - zjawisko, w którym następuje redukacja stanów kwantowych do stanów klasycznych.
\end{itemize}

\subsection{Komórka}
Komórka - najmniejszy, niezależny fragment rozgrywki. Tak samo jak w standardowej wersji gry w komórki wpisujemy znak (kółko lub krzyżyk). Różnica polega na tym, że w kwantowej wersji w jedno pole możemy wpisać wiele "małych" znaków (w szczególnym przypadku maksymalnie aż 8). Kiedy nastąpi \texttt{splątanie}, a następnie \texttt{zawalenie się} odpowiednie "małe znaki" zamienią się na stałe w "duże" znaki wówczas w każdym takim polu będzie znajdować się już tylko jeden znak.\\ \\

Każda komórka może znajdować się w jednym ze ściśle określonych stanów:
\begin{itemize}
    \item komórka pusta (początkowa sytuacja, gdy w komórce nie jest wpisany żaden znak)
    \item komórka wypełniona małymi znakami (sytuacja, gdy gracze wpisali już swoje znaki w komórkę, ale nie wystąpiło jeszcze \texttt{splątanie})
    \item komórka splątana (sytuacja, gdy właśnie nastąpiło \texttt{splątanie} i gracz musi wybrać, które znaki "małe" zamienią się na stałe w "duże")
    \item komórka po procesie "zawalenia się" (sytuacja, gdy nastąpiło \texttt{zawalenie się}; w danej komórce występuje tylko jeden znak, którego nie można zmienić)
\end{itemize}

\textbf{Uwaga!} Znaki "małe" to znaki, które są wpisywane przez graczy na początku każdej swojej tury. Znaki "duże" to znaki powstałe w procesie \texttt{zawalenia się}, których nie można już zmienić, a odpowiednie ich ułożenie zapewnia graczowi zwycięstwo.

\subsection{Plansza}
Plansza - rozbudowany element rozgrywki, który składa się z dziewięciu skorelowanych ze sobą komórek (ułożenie 3x3). W każdej turze gracz wpisuje dwa swoje znaki w komórki (\textbf{uwaga:} każdy ze znaków musi trafić do innej komórki!). W przypadku gdy wystąpi cykl (z jednej komórki można przejść do innej i wrócić z powrotem do pierwszej, ale inną ścieżką) gracz wybierze, który ze znaków "małych" zamieni się w duży. Takie działanie spowoduje "reakcję łańcuchową", ponieważ przerwany zostanie cykl i pozostałe komórki tworzące cykl wypełnią się odpowiednimi "dużymi" znakami. Poniżej prezentuję krótki scenariusz rozgrywki, gdzie nastąpi \texttt{splątanie} - w wyniku powstania cyklu:\\

\begin{figure}[h]
    \centering
    \includegraphics[width=0.2\textwidth]{plansza.png}
    \caption{plansza na początku rozgrywki (przed wykonaniem pierwszego ruchu)}
\end{figure}

\begin{figure}[h]
    \centering
    \includegraphics[width=0.2\textwidth]{plansza_1ruch.png}
    \caption{plansza po wykonanym pierwszym ruchu - ruch gracza x}
\end{figure}

\begin{figure}[h]
    \centering
    \includegraphics[width=0.2\textwidth]{plansza_2ruch.png}
    \caption{plansza po wykonanym drugim ruchu - ruch gracza y}
\end{figure}

\newpage

\begin{figure}[h]
    \centering
    \includegraphics[width=0.2\textwidth]{plansza_3ruch.png}
    \caption{plansza po wykonanym trzecim ruchu (splątanie) - ruch gracza x}
\end{figure}

Cykl tworzą komórki: lewa górna, środkowa górna oraz komórka centralna:

\begin{itemize}
    \item splątana jest komórka lewa górna z komórką środkową górną poprzez x1
    \item splątana jest komórka środkowa górna z komórką środkową centralną poprzez y2
    \item splątana jest komórka centralna z komórką lewą górną poprzez x3
\end{itemize}

Gracz przeciwny (tzn. nie ten który utworzył cykl) wybiera, które znaki zmieniają się w "duże". Przykładowo:

\begin{figure}[h]
    \centering
    \includegraphics[width=0.2\textwidth]{plansza_po_splataniu.png}
    \caption{plansza po wybraniu przez gracza y "dużych" znaków}
\end{figure}

\newpage

\subsection{Zakończenie rozgrywki}
Pomyślne zakończenie rozgrywki następuje tylko w dwóch przypadkach:\\
\begin{itemize}
    \item gdy któryś z graczy ułoży z trzech "dużych" znaków linie pionową/poziomą/po skosie - gracz, któremu udało się to jest zwycięzcą rozgrywki
    \item wypełnione zostaną wszystkie komórki planszy "dużymi" znakami, a żaden z graczy nie ułoży rzędu trzech znaków - żaden z graczy nie wygrywa
\end{itemize}

\textbf{UWAGA:} może wystąpić sytuacja szczególna, gdy dwóch graczy jednocześnie ułoży linie złożoną z trzech znaków! Wówczas zwycięzcą zostaje gracz, którego linia zawiera znak o mniejszym indeksie.

\begin{figure}[h]
    \centering
    \includegraphics[width=0.2\textwidth]{plansza_wygrana.png}
    \caption{plansza z utworzonymi liniami przez obu graczy}
\end{figure}
Najmniejszy indeks znaków w linii gracza x to 1, a w przypadku gracza y to 2. Zatem zwycięzcą rozgrywki zostaje gracz x.

\section{Sterowanie}
Sterowanie grą jest trywialne. Gracz by wykonać ruch jedyne co musi zrobić to podać z klawiatury w terminalu cyfrę odpowiadającą polu, które chce wybrać (cyfra z zakresu 0-8). Poniżej zaprezentowany jest schemat obrazujący przypisanie cyfr do konkretnych pól:
\begin{figure}[h]
    \centering
    \includegraphics[width=0.3\textwidth]{plansza_z_numeracja.png}

    \caption{plansza na początku rozgrywki}
\end{figure}

W przypadku wystąpienia splątania przeciwny gracz wybiera, które znaki zmienią się w "duże".

\newpage

\section{Komunikaty aplikacji}
W aplikacji komunikaty podzieliłem na dwa typy:
\begin{itemize}
    \item komunikaty informacyjne
    \item komunikaty błędów
\end{itemize}
Komunikaty informacyjne mają na celu pomaganie graczom rozgrywki (szczególnie tym, którzy nie są zaznajomieni z zasadami rozgrywki). W konsoli na bieżąco wyświetla się komunikat z informacją czyja jest tura. W przypadku wystąpienia "zawalenia się" (patrz podsekcja 3.3) gra poinformuje gracza o tym oraz podpowie, które znaki może wybrać.
\\ \\
Komunikaty błędów wystąpią w przypadku wykonania przez któregoś z graczy nieprawidłowego ruchu. Przykładowe nieprawidłowe zachowania gracza:
\begin{itemize}
    \item gracz próbuje wybrać numer pola, który nie występuje w planszy
    \item gracz próbuje wybrać dwa razy te same pole w jednej swojej turze
    \item gdy wystąpi splątanie gracz próbuje wybrać pole puste lub pole, które nie wchodzi w skład cyklu
\end{itemize}
W każdym z powyższych scenariuszy gracz zostanie poinformowany o nieprawidłowości w jego ruchu z wyjaśnieniem w jaki sposób wykonać poprawny ruch, błędy ruch nie zostanie wykonany, a gracz zostanie poproszony o wykonanie poprawnego działania.


\section{Testowanie programu}
Aplikacja będzie podlegała testowaniu na dwa sposoby - testy statyczne oraz testy jednostkowe. Testy statyczne - ręczna analiza pisanego kodu będzie wykonywana konsekwentnie w ciągu trwania całego projektu. Testy jednostkowe będą powstawać po wykonaniu pełnego, niezależnego fragmentu aplikacji (np. po stworzeniu klasy Gracz wraz z całą jej funkcjonalnością) zostaną napisane testy jednostkowe do danego fragmentu kodu. Na koniec projektu nastąpi weryfikacja pokrycia kodu testami jednostkowymi. W przypadku stwierdzenia, że jakiś fragment kodu/jakaś sytuacja nie jest sprawdzana zostaną napisane dodatkowe testy uzupełniające takowe braki. Testy jednostkowe zostaną napisane przy pomocy \textbf{Maven'a} przy użyciu biblioteki \texttt{JUnit}.

\newpage

\section{Bibliografia}
Do tworzenie powyższego dokumentu pomocne okazały się niżej wymienione źródła:

\begin{itemize}
    \item \url{https://en.wikipedia.org/wiki/Quantum\_tic-tac-toe}
    \item \url{http://qttt.rohanp.xyz/}
    \item \url{https://quantumfrontiers.com/2019/07/15/tiqtaqtoe/}
\end{itemize}

\end{document}
